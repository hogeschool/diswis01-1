\documentclass{beamer}
\usetheme[hideothersubsections]{HRTheme}
\usepackage{beamerthemeHRTheme}
\usepackage[T1]{fontenc}
\usepackage[utf8]{inputenc}
\usepackage{graphicx}
\usepackage[space]{grffile}
\usepackage{soul,xcolor}
\usepackage{listings}
\usepackage[simplified]{pgf -umlcd}
\usepackage{graphicx}
\usepackage{tabularx}
\lstset{language=C,
basicstyle=\ttfamily\footnotesize,
escapeinside={(*@}{@*)},
mathescape=true,
breaklines=true}
\lstset{
  literate={ï}{{\"i}}1
           {ì}{{\`i}}1
}

\title{Introduction to functional programming and lambda calculus}

\author{The INFDEV@HR Team}

\institute{Hogeschool Rotterdam \\ 
Rotterdam, Netherlands}

\date{}

\begin{document}
\maketitle
\SlideSection{Introduction}
\SlideSubSection{Lecture topics}
\begin{frame}[fragile]{\CurrentSection}
\begin{block}{\CurrentSubSection}
\begin{itemize}
\item Course introduction
\item Exam and practicum
\item Semantics of traditional programming languages
\item Basic lambda calculus

\end{itemize}

\end{block}


\end{frame}

\SlideSection{Course introduction}
\SlideSubSection{Course topics}
\begin{frame}[fragile]{\CurrentSection}
\begin{block}{\CurrentSubSection}
\begin{itemize}
\item We will discuss a completely new paradigm for expressing programs
\item This paradigm, functional programming, is based on different premises on computation
\item It gives guarantees of correctness in complex places, like parallelism or separation of concerns
\item It requires a radical conceptual shift in the way you think about programming

\end{itemize}

\end{block}


\end{frame}

\begin{frame}[fragile]{\CurrentSection}
\begin{block}{\CurrentSubSection}
\begin{itemize}
\item We will begin with a short discussion on traditional programming language \textbf{semantics}
\item We will then show the \textbf{lambda calculus}, which is the foundation for functional languages
\item \item We will then sho how such language can be used to express mathematical concepts such as:
 
\item \begin{exampleblock}{}
\begin{itemize}
\item Defining sums between numbers
\item Defining multiplications between numbers
\item Defining divisions between numbers
\item Defining Logarithms
\item etc..

\end{itemize}

\end{exampleblock}


 


\end{itemize}

\end{block}


\end{frame}

\SlideSection{Examination}
\SlideSubSection{Exam structure - idea}
\begin{frame}[fragile]{\CurrentSection}
\begin{block}{\CurrentSubSection}
\begin{itemize}
\item There is a theoretical exam, where you show understanding of the basic principles
\item There is a practical exam, where you show understanding of their concrete applications

\end{itemize}

\end{block}


\end{frame}

\SlideSection{Semantics of traditional programming languages}
\SlideSubSection{Semicolon and interference}
\begin{frame}[fragile]{\CurrentSection}
\begin{exampleblock}{}
\begin{itemize}
\item Traditional, imperative programming languages are based on sharing memory through instructions
\item This means that subsequent instructions are not independent from each other
\item Any function call makes use of the available memory

\end{itemize}

\end{exampleblock}

 

\end{frame}

\begin{frame}[fragile]{\CurrentSection}
\begin{exampleblock}{}
For example, consider the semantic rules that describe the working of ``;''
\end{exampleblock}

 
\begin{exampleblock}{}
First we run $s_1$ with the initial memory, then we run $s_2$ with the modified memory.
\end{exampleblock}

 
\pause 
$$\frac{\ \langle s_1,S,H \rangle \rightarrow \langle S_1,H_1 \rangle\wedge\ \langle s_2,S_1,H_1 \rangle \rightarrow \langle S_2,H_2 \rangle}{\langle (s_1;s_2),S,H \rangle \rightarrow \langle S_2,H_2 \rangle}$$
 

\end{frame}

\begin{frame}[fragile]{\CurrentSection}
\begin{exampleblock}{}
What does ``\textit{first we run $s_1$ with the initial memory, then we run $s_2$ with the modified memory}'' imply?.
\end{exampleblock}

 
\pause 
\begin{exampleblock}{}
\begin{itemize}
\item The same instructions, executed at different moments, will produce \textbf{different results}.
\item Change the order of some method calls, and some weird dependence might cause bugs or break things.

\end{itemize}

\end{exampleblock}

 

\end{frame}

\SlideSubSection{Goals}
\begin{frame}[fragile]{\CurrentSection}
\begin{block}{\CurrentSubSection}
\begin{itemize}
\item Our goal is to ensure that behaviour of code is consistent.
\item Change the order of some method calls, and the results remain the same.
\item This makes it easier to test, parallelize, and in general ensure correctness.

\end{itemize}

\end{block}


\end{frame}

\begin{frame}[fragile]{\CurrentSection}
\begin{exampleblock}{}
\textit{How do we achieve this?}
\end{exampleblock}

 
\pause 
\begin{exampleblock}{}
We give (shared) memory up: every piece of code is a function which output only depends on input.
\end{exampleblock}

 
\begin{exampleblock}{}
This very important property is called \textbf{referential transparency}.
\end{exampleblock}

 

\end{frame}

\SlideSection{Basic lambda calculus}
\SlideSubSection{Introduction}
\begin{frame}[fragile]{\CurrentSection}
\begin{block}{\CurrentSubSection}
\begin{itemize}
\item The (basic) lambda calculus is an alternative mechanism to Turing Machines and the Von Neumann architecture.
\item It is very different, but has equivalent expressive power.
\item It is the foundation of all functional programming languages.

\end{itemize}

\end{block}


\end{frame}

\SlideSubSection{Substitution principle}
\begin{frame}[fragile]{\CurrentSection}
\begin{block}{\CurrentSubSection}
\begin{itemize}
\item The (basic) lambda calculus is truly tiny when compared with its power.
\item It is based on the substitution principle: calling a function with some parameters returns the function body with the variables replaced.
\item There is no memory and no program counter: all we need to know is stored inside the body of the program itself.

\end{itemize}

\end{block}


\end{frame}

\SlideSubSection{Grammar}
\begin{frame}[fragile]{\CurrentSection}
\begin{exampleblock}{}
A lambda calculus program (just \textit{program} from now on) is made up of three syntactic elements:
\end{exampleblock}

 
\begin{exampleblock}{}
\begin{itemize}
\item Variables: $x$, $y$, ...
\item Abstractions (function declarations with one parameter): $\lambda x\rightarrow t$ where $x$ is a variable and $t$ is the function body (a program).
\item Applications (function calls with one argument): $t\ u$ where $t$ is the function being called (a program) and $u$ is its argument (another program).

\end{itemize}

\end{exampleblock}

 

\end{frame}

\begin{frame}[fragile]{\CurrentSection}
\begin{exampleblock}{}
A simple example would be the identity function, which just returns whatever it gets as input
\end{exampleblock}

 
\lstset{basicstyle=\ttfamily\small}\lstset{numbers=none}\lstset{language=ML}\begin{lstlisting}
($\lambda$x$\rightarrow$x)
\end{lstlisting}
 

\end{frame}

\begin{frame}[fragile]{\CurrentSection}
\begin{exampleblock}{}
We can call this function with a variable as argument, by writing:
\end{exampleblock}

 
\lstset{basicstyle=\ttfamily\small}\lstset{numbers=none}\lstset{language=ML}\begin{lstlisting}
(($\lambda$x$\rightarrow$x) v)
\end{lstlisting}
 

\end{frame}

\SlideSubSection{Beta reduction}
\begin{frame}[fragile]{\CurrentSection}
\begin{exampleblock}{}
A lambda calculus program is computed by replacing lambda abstractions applied to arguments with the body of the lambda abstraction with the argument instead of the lambda parameter:
\end{exampleblock}

 
\pause 
$$\frac{}{(\lambda x \rightarrow t)\ u  \rightarrow_\beta t [ x \mapsto u ]}$$
 
\begin{exampleblock}{}
$t [ x \mapsto u ]$ means that we change variable $x$ with $u$ within $t$
\end{exampleblock}

 

\end{frame}

\begin{frame}[fragile]{\CurrentSection}
\lstset{basicstyle=\ttfamily\small}\lstset{numbers=none}\lstset{language=ML}\begin{lstlisting}
(($\lambda$x$\rightarrow$x) v)
\end{lstlisting}
\pause\lstset{language=ML}\begin{lstlisting}
(*@\colorbox{orange}{(($\lambda$x$\rightarrow$x) v)}@*)
\end{lstlisting}

\end{frame}

\begin{frame}[fragile]{\CurrentSection}
\lstset{basicstyle=\ttfamily\small}\lstset{numbers=none}\lstset{language=ML}\begin{lstlisting}
(*@\colorbox{orange}{(($\lambda$x$\rightarrow$x) v)}@*)
\end{lstlisting}
\pause\lstset{language=ML}\begin{lstlisting}
(*@\colorbox{yellow}{v}@*)
\end{lstlisting}

\end{frame}

\begin{frame}[fragile]{\CurrentSection}
\begin{exampleblock}{}
Multiple applications where the left-side is not a lambda abstraction are solved in a left-to-right fashion:
\end{exampleblock}

 
\pause 
$$\frac{\ t \rightarrow_\beta t'\wedge\ u \rightarrow_\beta u'\wedge\ t'\ u' \rightarrow_\beta v}{t\ u \rightarrow_\beta v}$$
 

\end{frame}

\begin{frame}[fragile]{\CurrentSection}
\begin{exampleblock}{}
Variables cannot be further reduced, that is they stay the same:
\end{exampleblock}

 
\pause 
$$\frac{}{x \rightarrow_\beta x}$$
 

\end{frame}

\SlideSubSection{Multiple parameters}
\begin{frame}[fragile]{\CurrentSection}
\begin{exampleblock}{}
We can encode functions with multiple parameters by nesting lambda abstractions:
\end{exampleblock}

 
\lstset{basicstyle=\ttfamily\small}\lstset{numbers=none}\lstset{language=ML}\begin{lstlisting}
($\lambda$x y$\rightarrow$(x y))
\end{lstlisting}
 

\end{frame}

\begin{frame}[fragile]{\CurrentSection}
\begin{exampleblock}{}
The parameters are then given one at a time:
\end{exampleblock}

 
\lstset{basicstyle=\ttfamily\small}\lstset{numbers=none}\lstset{language=ML}\begin{lstlisting}
((($\lambda$x y$\rightarrow$(x y)) A) B)
\end{lstlisting}
 

\end{frame}

\begin{frame}[fragile]{\CurrentSection}
\lstset{basicstyle=\ttfamily\small}\lstset{numbers=none}\lstset{language=ML}\begin{lstlisting}
((($\lambda$x y$\rightarrow$(x y)) A) B)
\end{lstlisting}
\pause\lstset{language=ML}\begin{lstlisting}
((*@\colorbox{orange}{(($\lambda$x y$\rightarrow$(x y)) A)}@*) B)
\end{lstlisting}

\end{frame}

\begin{frame}[fragile]{\CurrentSection}
\lstset{basicstyle=\ttfamily\small}\lstset{numbers=none}\lstset{language=ML}\begin{lstlisting}
((*@\colorbox{orange}{(($\lambda$x y$\rightarrow$(x y)) A)}@*) B)
\end{lstlisting}
\pause\lstset{language=ML}\begin{lstlisting}
(($\lambda$y$\rightarrow$((*@\colorbox{yellow}{A}@*) y)) B)
\end{lstlisting}

\end{frame}

\begin{frame}[fragile]{\CurrentSection}
\lstset{basicstyle=\ttfamily\small}\lstset{numbers=none}\lstset{language=ML}\begin{lstlisting}
(($\lambda$y$\rightarrow$(A y)) B)
\end{lstlisting}
\pause\lstset{language=ML}\begin{lstlisting}
(*@\colorbox{orange}{(($\lambda$y$\rightarrow$(A y)) B)}@*)
\end{lstlisting}

\end{frame}

\begin{frame}[fragile]{\CurrentSection}
\lstset{basicstyle=\ttfamily\small}\lstset{numbers=none}\lstset{language=ML}\begin{lstlisting}
(*@\colorbox{orange}{(($\lambda$y$\rightarrow$(A y)) B)}@*)
\end{lstlisting}
\pause\lstset{language=ML}\begin{lstlisting}
(A (*@\colorbox{yellow}{B}@*))
\end{lstlisting}

\end{frame}

\SlideSection{Closing up}
\SlideSubSection{Example executions of (apparently) nonsensical programs}
\begin{frame}[fragile]{\CurrentSection}
\begin{block}{\CurrentSubSection}
\begin{itemize}
\item We will now exercise with the execution of various lambda programs.
\item Try to guess what the result of these programs is, and then we shall see what would have happened.

\end{itemize}

\end{block}


\end{frame}

\begin{frame}[fragile]{\CurrentSection}
\begin{exampleblock}{}
\textit{What is the result of this program execution?}
\end{exampleblock}

 
\lstset{basicstyle=\ttfamily\small}\lstset{numbers=none}\lstset{language=ML}\begin{lstlisting}
((($\lambda$x y$\rightarrow$(x y)) ($\lambda$z$\rightarrow$(z z))) A)
\end{lstlisting}
 

\end{frame}

\begin{frame}[fragile]{\CurrentSection}
\lstset{basicstyle=\ttfamily\small}\lstset{numbers=none}\lstset{language=ML}\begin{lstlisting}
((($\lambda$x y$\rightarrow$(x y)) ($\lambda$z$\rightarrow$(z z))) A)
\end{lstlisting}
\pause\lstset{language=ML}\begin{lstlisting}
((*@\colorbox{orange}{(($\lambda$x y$\rightarrow$(x y)) ($\lambda$z$\rightarrow$(z z)))}@*) A)
\end{lstlisting}

\end{frame}

\begin{frame}[fragile]{\CurrentSection}
\lstset{basicstyle=\ttfamily\small}\lstset{numbers=none}\lstset{language=ML}\begin{lstlisting}
((*@\colorbox{orange}{(($\lambda$x y$\rightarrow$(x y)) ($\lambda$z$\rightarrow$(z z)))}@*) A)
\end{lstlisting}
\pause\lstset{language=ML}\begin{lstlisting}
(($\lambda$y$\rightarrow$((*@\colorbox{yellow}{($\lambda$z$\rightarrow$(z z))}@*) y)) A)
\end{lstlisting}

\end{frame}

\begin{frame}[fragile]{\CurrentSection}
\lstset{basicstyle=\ttfamily\small}\lstset{numbers=none}\lstset{language=ML}\begin{lstlisting}
(($\lambda$y$\rightarrow$(($\lambda$z$\rightarrow$(z z)) y)) A)
\end{lstlisting}
\pause\lstset{language=ML}\begin{lstlisting}
(*@\colorbox{orange}{(($\lambda$y$\rightarrow$(($\lambda$z$\rightarrow$(z z)) y)) A)}@*)
\end{lstlisting}

\end{frame}

\begin{frame}[fragile]{\CurrentSection}
\lstset{basicstyle=\ttfamily\small}\lstset{numbers=none}\lstset{language=ML}\begin{lstlisting}
(*@\colorbox{orange}{(($\lambda$y$\rightarrow$(($\lambda$z$\rightarrow$(z z)) y)) A)}@*)
\end{lstlisting}
\pause\lstset{language=ML}\begin{lstlisting}
(($\lambda$z$\rightarrow$(z z)) (*@\colorbox{yellow}{A}@*))
\end{lstlisting}

\end{frame}

\begin{frame}[fragile]{\CurrentSection}
\lstset{basicstyle=\ttfamily\small}\lstset{numbers=none}\lstset{language=ML}\begin{lstlisting}
(($\lambda$z$\rightarrow$(z z)) A)
\end{lstlisting}
\pause\lstset{language=ML}\begin{lstlisting}
(*@\colorbox{orange}{(($\lambda$z$\rightarrow$(z z)) A)}@*)
\end{lstlisting}

\end{frame}

\begin{frame}[fragile]{\CurrentSection}
\lstset{basicstyle=\ttfamily\small}\lstset{numbers=none}\lstset{language=ML}\begin{lstlisting}
(*@\colorbox{orange}{(($\lambda$z$\rightarrow$(z z)) A)}@*)
\end{lstlisting}
\pause\lstset{language=ML}\begin{lstlisting}
((*@\colorbox{yellow}{A}@*) (*@\colorbox{yellow}{A}@*))
\end{lstlisting}

\end{frame}

\begin{frame}[fragile]{\CurrentSection}
\begin{exampleblock}{}
\textit{What is the result of this program execution? Watch out for the scope of the two ``x'' variables!}
\end{exampleblock}

 
\lstset{basicstyle=\ttfamily\small}\lstset{numbers=none}\lstset{language=ML}\begin{lstlisting}
((($\lambda$x x$\rightarrow$(x x)) A) B)
\end{lstlisting}
 

\end{frame}

\begin{frame}[fragile]{\CurrentSection}
\lstset{basicstyle=\ttfamily\small}\lstset{numbers=none}\lstset{language=ML}\begin{lstlisting}
((($\lambda$x x$\rightarrow$(x x)) A) B)
\end{lstlisting}
\pause\lstset{language=ML}\begin{lstlisting}
((*@\colorbox{orange}{(($\lambda$x x$\rightarrow$(x x)) A)}@*) B)
\end{lstlisting}

\end{frame}

\begin{frame}[fragile]{\CurrentSection}
\lstset{basicstyle=\ttfamily\small}\lstset{numbers=none}\lstset{language=ML}\begin{lstlisting}
((*@\colorbox{orange}{(($\lambda$x x$\rightarrow$(x x)) A)}@*) B)
\end{lstlisting}
\pause\lstset{language=ML}\begin{lstlisting}
(($\lambda$x$\rightarrow$(x x)) B)
\end{lstlisting}

\end{frame}

\begin{frame}[fragile]{\CurrentSection}
\lstset{basicstyle=\ttfamily\small}\lstset{numbers=none}\lstset{language=ML}\begin{lstlisting}
(($\lambda$x$\rightarrow$(x x)) B)
\end{lstlisting}
\pause\lstset{language=ML}\begin{lstlisting}
(*@\colorbox{orange}{(($\lambda$x$\rightarrow$(x x)) B)}@*)
\end{lstlisting}

\end{frame}

\begin{frame}[fragile]{\CurrentSection}
\lstset{basicstyle=\ttfamily\small}\lstset{numbers=none}\lstset{language=ML}\begin{lstlisting}
(*@\colorbox{orange}{(($\lambda$x$\rightarrow$(x x)) B)}@*)
\end{lstlisting}
\pause\lstset{language=ML}\begin{lstlisting}
((*@\colorbox{yellow}{B}@*) (*@\colorbox{yellow}{B}@*))
\end{lstlisting}

\end{frame}

\begin{frame}[fragile]{\CurrentSection}
\begin{exampleblock}{}
The first ``x'' gets replaced with ``A'', but the second ``x'' shadows it!
\end{exampleblock}

 
\lstset{basicstyle=\ttfamily\small}\lstset{numbers=none}\lstset{language=ML}\begin{lstlisting}
((($\lambda$x x$\rightarrow$(x x)) A) B)
\end{lstlisting}
 

\end{frame}

\begin{frame}[fragile]{\CurrentSection}
\begin{exampleblock}{}
A better formulation, less ambiguous, would turn:
\end{exampleblock}

 
\lstset{basicstyle=\ttfamily\small}\lstset{numbers=none}\lstset{language=ML}\begin{lstlisting}
((($\lambda$x x$\rightarrow$(x x)) A) B)
\end{lstlisting}
 
\begin{exampleblock}{}
...into:
\end{exampleblock}

 
\lstset{basicstyle=\ttfamily\small}\lstset{numbers=none}\lstset{language=ML}\begin{lstlisting}
((($\lambda$y x$\rightarrow$(x x)) A) B)
\end{lstlisting}
 

\end{frame}

\begin{frame}[fragile]{\CurrentSection}
\lstset{basicstyle=\ttfamily\small}\lstset{numbers=none}\lstset{language=ML}\begin{lstlisting}
((($\lambda$y x$\rightarrow$(x x)) A) B)
\end{lstlisting}
\pause\lstset{language=ML}\begin{lstlisting}
((*@\colorbox{orange}{(($\lambda$y x$\rightarrow$(x x)) A)}@*) B)
\end{lstlisting}

\end{frame}

\begin{frame}[fragile]{\CurrentSection}
\lstset{basicstyle=\ttfamily\small}\lstset{numbers=none}\lstset{language=ML}\begin{lstlisting}
((*@\colorbox{orange}{(($\lambda$y x$\rightarrow$(x x)) A)}@*) B)
\end{lstlisting}
\pause\lstset{language=ML}\begin{lstlisting}
(($\lambda$x$\rightarrow$(x x)) B)
\end{lstlisting}

\end{frame}

\begin{frame}[fragile]{\CurrentSection}
\lstset{basicstyle=\ttfamily\small}\lstset{numbers=none}\lstset{language=ML}\begin{lstlisting}
(($\lambda$x$\rightarrow$(x x)) B)
\end{lstlisting}
\pause\lstset{language=ML}\begin{lstlisting}
(*@\colorbox{orange}{(($\lambda$x$\rightarrow$(x x)) B)}@*)
\end{lstlisting}

\end{frame}

\begin{frame}[fragile]{\CurrentSection}
\lstset{basicstyle=\ttfamily\small}\lstset{numbers=none}\lstset{language=ML}\begin{lstlisting}
(*@\colorbox{orange}{(($\lambda$x$\rightarrow$(x x)) B)}@*)
\end{lstlisting}
\pause\lstset{language=ML}\begin{lstlisting}
((*@\colorbox{yellow}{B}@*) (*@\colorbox{yellow}{B}@*))
\end{lstlisting}

\end{frame}

\begin{frame}[fragile]{\CurrentSection}
\begin{exampleblock}{}
\textit{What is the result of this program execution? Is there even a result?}
\end{exampleblock}

 
\lstset{basicstyle=\ttfamily\small}\lstset{numbers=none}\lstset{language=ML}\begin{lstlisting}
(($\lambda$x$\rightarrow$(x x)) ($\lambda$x$\rightarrow$(x x)))
\end{lstlisting}
 

\end{frame}

\begin{frame}[fragile]{\CurrentSection}
\lstset{basicstyle=\ttfamily\small}\lstset{numbers=none}\lstset{language=ML}\begin{lstlisting}
(($\lambda$x$\rightarrow$(x x)) ($\lambda$x$\rightarrow$(x x)))
\end{lstlisting}
\pause\lstset{language=ML}\begin{lstlisting}
(*@\colorbox{orange}{(($\lambda$x$\rightarrow$(x x)) ($\lambda$x$\rightarrow$(x x)))}@*)
\end{lstlisting}

\end{frame}

\begin{frame}[fragile]{\CurrentSection}
\lstset{basicstyle=\ttfamily\small}\lstset{numbers=none}\lstset{language=ML}\begin{lstlisting}
(*@\colorbox{orange}{(($\lambda$x$\rightarrow$(x x)) ($\lambda$x$\rightarrow$(x x)))}@*)
\end{lstlisting}
\pause\lstset{language=ML}\begin{lstlisting}
((*@\colorbox{yellow}{($\lambda$x$\rightarrow$(x x))}@*) (*@\colorbox{yellow}{($\lambda$x$\rightarrow$(x x))}@*))
\end{lstlisting}

\end{frame}

\begin{frame}[fragile]{\CurrentSection}
\lstset{basicstyle=\ttfamily\small}\lstset{numbers=none}\lstset{language=ML}\begin{lstlisting}
(($\lambda$x$\rightarrow$(x x)) ($\lambda$x$\rightarrow$(x x)))
\end{lstlisting}
\pause\lstset{language=ML}\begin{lstlisting}
(*@\colorbox{orange}{(($\lambda$x$\rightarrow$(x x)) ($\lambda$x$\rightarrow$(x x)))}@*)
\end{lstlisting}

\end{frame}

\begin{frame}[fragile]{\CurrentSection}
\lstset{basicstyle=\ttfamily\small}\lstset{numbers=none}\lstset{language=ML}\begin{lstlisting}
(*@\colorbox{orange}{(($\lambda$x$\rightarrow$(x x)) ($\lambda$x$\rightarrow$(x x)))}@*)
\end{lstlisting}
\pause\lstset{language=ML}\begin{lstlisting}
((*@\colorbox{yellow}{($\lambda$x$\rightarrow$(x x))}@*) (*@\colorbox{yellow}{($\lambda$x$\rightarrow$(x x))}@*))
\end{lstlisting}

\end{frame}

\begin{frame}[fragile]{\CurrentSection}
\lstset{basicstyle=\ttfamily\small}\lstset{numbers=none}\lstset{language=ML}\begin{lstlisting}
(($\lambda$x$\rightarrow$(x x)) ($\lambda$x$\rightarrow$(x x)))
\end{lstlisting}
 
\begin{exampleblock}{}
It never ends! Like a \texttt{while true: ..}!
\end{exampleblock}

 

\end{frame}

\SlideSubSection{Crazy teachers tormenting poor students, or ``where are my integers?''}
\begin{frame}[fragile]{\CurrentSection}
\begin{exampleblock}{}
Ok, I know what you are all thinking: what is this for sick joke? This is no real programming language!
\end{exampleblock}

 
\begin{exampleblock}{}
\begin{itemize}
\item We have some sort of functions and function calls
\item We do not have booleans and \textttt{if}'s
\item We do not have integers and arithmetic operators
\item We do not have a lot of things!

\end{itemize}

\end{exampleblock}

 

\end{frame}

\SlideSubSection{Surprise!}
\begin{frame}[fragile]{\CurrentSection}
\begin{block}{\CurrentSubSection}
With nothing but lambda programs we will show how to build all of these features and more.
\end{block}


\end{frame}

\SlideSubSection{Stay tuned.}
\begin{frame}[fragile]{\CurrentSection}
\begin{block}{\CurrentSubSection}
This will be a marvelous voyage.
\end{block}


\end{frame}


\begin{frame}{This is it!}
\center
\fontsize{18pt}{7.2}\selectfont
The best of luck, and thanks for the attention!
\end{frame}

\end{document}